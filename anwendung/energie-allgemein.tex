\label{energie-allgemein}
\section{Erste Abschätzungen mithilfe der allgemeinen Energieerhaltung}

Bei vielen Problemen in der Mechanik ist es ausreichend oder zumindest hilfreich zu versuchen, das Problem mithilfe der Energieerhaltung zu formulieren, um eine erste Abschätzung vornehmen zu können. 

\paragraph{Energie des Wurfarmes.}
So auch hier: Zunächst bestimmen wir die Gesamtenergie, die wir dem Trebuchet zuführen:
\begin{align}
\label{gesammtenergie}
E_{ges} &= m_G g h_0 %\emph{Anfängliche, potentielle Energie des Trebuchets}
\end{align}
Dabei ist $h_0$ die Höhe und $m_G$ die Masse des Gegengewichtes. Wenn wir davon ausgehen, dass das Geschoss in dem Moment freigegeben wird, in dem das Gegengewicht seine tiefste Stelle erreicht (also wenn $h=0$ gilt), wird diese Energie auch bestmöglich genutzt.
Man könnte jetzt (vorschnell) schlussfolgern, dass die Reichweite des Trebuchets direkt von $E_{ges}$ abhinge, also $E_{ges}=E_P$ sei. Jedoch schwingt der Wurfarm nach Abwurf des Geschosses weiter. Dafür ist natürlich auch Energie notwendig, die dann nicht mehr dem Geschoss zur Verfügung stehen kann.
Natürlich gibt es auch keine lineare Abhängigkeit, also etwa $E_P=\frac{m_P}{{m_{Trebuchet}}} E_{ges}$, denn der Wurfarm hat an verschiedenen Stellen eine unterschiedliche Geschwindigkeit und deswegen auch eine unterschiedlich hohe, kinetische Energie.

\paragraph{Energie des Geschosses.}
Die Reichweite $s$ des Trebuchets hängt hauptsächlich von der Anfangsgeschwindigkeit und so von der anfänglichen, kinetischen Energie des Geschosses ab:
\begin{align}
\label{wurfweite}
s &= \frac{v_{0, P}^2 \sin{2\varphi_P}} {g} \\%\emph{Wurfweite des schiefen Wurfes (ohne Luftwiederstand)}\\
E_P &= \frac{m_P v_P^2}{2} %\emph{Kinetische Energie des Geschosses}
\end{align}
Wenn man die Luftreibung vernachlässigt und den Abwurfwinkel $\varphi_P=\frac{\pi}{4}$ setzt, um die maximale Reichweite zu erzielen (ein Winkel von $0$ entspricht der Waagerechten), ergibt sich durch Äquivalenzumformung sofort die Reichweite in Abhängigkeit von der Energie:
\begin{align}
s(E) &= \frac{2 E_P}{ m_P}\frac{\sin{2\varphi_P}}{g}%\emph{Reichweite des Trebuchets}
\end{align}
Damit ist klar, dass für die reale Reichweite $s_{real}$ in jedem Fall
\begin{align}
s_{real}&<s(E_{ges})
\end{align}
gelten muss. Mehr Informationen können mit der einfachen Energieerhaltung noch nicht gewonnen werden (die Verbesserung wird in (\ref{energie-rot}) vorgenommen).