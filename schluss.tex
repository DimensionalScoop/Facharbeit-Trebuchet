\chapter{Schluss}
\section{Reflexion}
Ich wäre gerne noch genauer auf die Berechnung des Trägheitsmoments eingegangen, z.B. in Form einer ausführlichen Herleitung. Auch wäre eine Herleitung zum Steiner'sche Verschiebungssatz mithilfe von (\ref{vol-I}) wünschenswert gewesen. Ich habe darauf aber verzichtet, da diese Herleitungen mathematischer Natur sind und die Länge dieser Facharbeit noch weiter gesteigert hätten.

Ein weiterer Punkt ist der Verzicht auf die Definition der Einheiten von $I$, $\vec M$ und $\vec L$. Dies geschah vor allen deswegen, weil sie für die Berechnungen am Trebuchet keine Rolle spielen, da sie sich am Ende herauskürzen, und weil diese Einheiten tendenziell eher schwierig zu fassen, d.h. vorzustellen sind. Es ist also nicht zweckdienlich, diese hier zu definieren.

Ich habe auch weitestgehend lange Äquivalenzumformungen vermieden, um die Facharbeit kurz und übersichtlich zu halten. Äquivalenzumformungen sind schließlich rein mathematisch und im Zweifelsfall am Computer oder auf dem Papier schnell und einfach nachzuvollziehen.

Ferner  hätte ich gerne noch die Reibungsverluste an der Achse mit in die Formel aufgenommen, aber der Reibungskoeffizienten von PVC gegen PVC scheint nur sehr schwierig zu finden zu sein. Das gleiche gilt für die \glqq Federhärte\grqq \ der Rohre.

\section{Ausblick}
Mit den diskutierten Formeln kann man natürlich nicht nur das Experiment überprüfen, sondern auch verbessern. Z.B. mit einem Funktionsplotter lässt sich die Reichweite in Abhängigkeit verschiedener Parameter trotz der komplexen Formel gut darstellen und so das bereits gebaute Trebuchet optimieren.

Das Schmieren der Drehachse oder der Einbau eines Kugellagers könnten ferner die Reichweite zusätzlich erhöhen. Eine andere Möglichkeit ist es, das Gegengewicht nicht nur mit der Schnur am Hebel zu befestigen, sondern auch diese über eine Evolvente (wie etwa über einen Zahn eines Zahnrads) abzurollen, um den Hebel beim Fallen des Gegengewichts zu vergrößern, um die sich verringernde Kraft senkrecht zum Hebelarm zu kompensieren.

Eine sehr einfache Möglichkeit ist es, das Gegengewicht zu vergrößern: Bei meinen Experimenten habe ich mit bis zu 18 Kg getestet, jedoch sind für weitere Massesteigerungen dickere Rohre notwendig, um eine zu große Belastung an den Übergangsstellen, die flexibel sein müssen, zu vermeiden. Es ist auch möglich, den Wurfarm ganz oder teilweise durch kohlefaserverstärktes Drachengestänge zu ersetzen, um das Trägheitsmoment zu verringern (ein Halbieren der Hebelmasse hätte eine 1.8xFache Steigerung der Reichweite zur Folge).

Aber auch die Formel selbst lässt sich noch verbessern: Durch das Einbeziehen von Reibung an der Achse kann die Genauigkeit noch erhöht werden. Mithilfe einer numerischen Analyse kann auch der Luftwiderstand des Hebels und ohne die Vereinfachung gerechnet werden. Natürlich ist dafür dann aber ein Computer notwendig.